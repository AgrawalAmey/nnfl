\documentclass{article} % For LaTeX2e
\usepackage{graphicx,caption}
\usepackage{multicol}
\usepackage{hyperref}
\usepackage{amsmath}
\usepackage{algorithm}
\usepackage{amssymb}
\usepackage[noend]{algpseudocode}
\hypersetup{
    colorlinks=true,
    linkcolor=blue,
    filecolor=magenta,
    urlcolor=cyan,
}



\def\BState{\State\hskip-\ALG@thistlm}
\newcommand{\R}{\mathbb{R}}

\begin{document}

\begin{titlepage}
    \begin{center}
        \vspace*{1cm}

        \begin{huge}
            \textbf{Neural Networks and Fuzzy Logic}
        \end{huge}
        \LARGE{Overview of programming assignments}

        \vspace{4cm}

        \textbf{The course team}\\
        \today

        \vspace{2.5cm}

        \includegraphics[width=0.4\textwidth]{logo}
        \vspace{1cm}

        Birla Institute of Technology and Science, Pilani\\
        \vspace{0.5cm}
    \end{center}
\end{titlepage}


\tableofcontents

\vspace{10cm}

\section{Introduction}

The five programming assignments are designed to 
to reinforce the understanding of course material 
and their applications.

\section{Timeline}

The following represents an approximate timeline for the
programming components.

\vspace{0.5cm}

\begin{tabular}{ | p{10em} || p{25em} | }
\hline
8th August & Pre-course survey. \\
\hline
12th August & Hands-on-coding session 0: Introduction to
             Python and Numpy.\\
\hline
14th August & Assignment 0 release: Introduction to Matrix operations
                            and Fuzzy logic. \\
\hline
23rd August & Assignment 0 Submission. \\
\hline
29th August & Hand-on-coding session 1: Plotting with Matplotlib and k-means clustering. \\
\hline
2nd September & Assignment 1 release: Fuzzy Logic and Genetic Algorithms. \\
\hline
11th September & Assignment 1 Submission.\\
\hline
14th September & Hand-on-coding Session 2: Introduction to Scikit-Learn for data pre-processing and visualisation. \\
\hline
16th September & Assignment 2 release: RBF Networks, SVM and Multi-layer perceptron.\\
\hline
25th September & Assignment 2 Submission.\\
\hline
3rd October & Mid-Semester survey.\\
\hline
17th October & Hand-on-coding session 3: Hassle-free set-up of deep learning frameworks with DL-docker, Introduction to Pytorch.\\
\hline
17th October & Assignment 3 release: Exploring activation functions and variations of gradient descent,
                                     Introduction to high-level machine learning libraries. \\
\hline
26rd October & Assignment 3 Submission. \\
\hline
26th October & Hand-on-coding session 4: Scikit-Learn and Scikit-fuzzy by examples.\\
\hline
7th November & Hand-on-coding session 5: Pytorch by examples.\\
\hline
7th November & Open hours: Assignment 4 topic selection help.\\
\hline
7th November & Assignment 4 release: Guided project.\\
\hline
14th November & Assignment 4:Stage 0 evaluation.\\
\hline
21-23nd November & Assignment 4:Stage 1 evaluation.\\
\hline
26th November & End-Semester survey.\\
\hline
\end{tabular}

\section{Evaluation scheme}
The first four assignments will be for 15 marks each and the final assignment would 
carry 45 marks out of the course total of 300 marks.

\section{Details}

\subsection{Hands-on-coding session 0}

The first session would provide an introduction to Python, Numpy and jupyter notebooks.

\subsection{Assignment 0}

Assignment 0 would start with some practice problems on python basics and Numpy, so as to get students comfortable with matrix operations.
The problems shall be focused on data handling in python and core Numpy concepts
including slicing, fancy-indexing and masked index.

The later half of the assignment would deal with fuzzy set operations, the functions written in the first assignment shall be
used for more implementation of the fuzzy logic algorithms in the next assignment.

\subsection{Hands-on-coding session 1}

Introduction of Matplotlib and demonstration of different examples of loading and plotting data in python will be 
the main agenda of this session.

\subsection{Assignment 1}

The first half of assignment 1 will be concerned with implementing a fuzzy c-means clustering and defuzzification algorithms.

Formulating a problem for genetic algorithm could be tricky and shall be the centre of our focus.
Students will have to model solution for a real world problem and compete in Kaggle like competition. 

\subsection{Hands-on-coding session 2}

Data pre-processing is an crucial step to make any learning algorithm work
and hence we shall cover vector representations for different kind of data
and normalization schemes. In the later half of session will
provide a brief introduction to Scikit-learn. 

\subsection{Assignment 2}

Students would have to model one common problem throughout using SVM, RBF Nets and 
Multi-layer perceptron and perform comparative analysis of the three algorithms.

This assignments will also have a live scoreboard like the previous one.

\subsection{Hands-on-coding session 3}

In this session, we shall provide a gentle
introduction to pytorch and auto-differentiation libraries.

\subsection{Assignment 3}

Activation functions and decent algorithm are possibly the
most important meta parameter in a neural network. The first half of the
assignment would deal with implementing various activation functions
and variations of gradient descent in numpy (building over the code
in assignment 2) and analysing it's results.

In the later half, we shall have a modelling problem which students
would have to code in pytorch. Again, we could have a scoreboard.
This exercise would be instrumental towards the final assignment.


\subsection{Hands-on-coding session 4}

In this session we would explore some examples using
Scikit-Learn and Scikit-fuzzy. This session is meant to facilitate 
final assignment.

\subsection{Hands-on-coding session 5}

The sixth and final hands-on coding session would also be on similar
lines as that of the previous one except it would focus on applying
pytorch on real world problems.

\subsection{Assignment 4}

The final assignment will be special since it will carry thrice the
weightage of other assignments. Students would have to work in teams 
for this assignment. Every team will be assigned a specific paper 
based on their prior knowledge and interests.

The first stage of evaluation would test the understanding of the
paper. While the second stage would involve submission of code and Viva. 

\end{document}
